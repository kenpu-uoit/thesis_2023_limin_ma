% Lucene and inverted list
Apache Lucene is an Java-based open-source search engine library that offers full-text indexing and search features.
Lucene was initially released in 1999 and joined Apache Software Foundation in 2001 \cite{McCandless2010}.
According to Lucene's features webpage \footnote{Lucene Features: \url{https://lucene.apache.org/core/features.html}}, 
its indexing implementation is scalable and effcient,  which can index over 800GB/hour with only 1MB heap size.
It also implements accurate and efficient search algorithms. Under the hood, Lucene uses inverted index.
Inverted index is an indexing data structure that maps a word to documents. This allows Lucene to lookup keywords and return documents efficiently.

% String matching
When we search documents, exact match may not be the ideal approach. A search keyword could be spelled slightly differently, e.g. "color" vs "colour".
A user may misspell a keyword. In addition, we may want to search documents based on a keyword and all its variations, e.g. "Canada" and "Canadian".
Therefore, fuzzy string matching plays an important role in text search. There are various methods to match string approximately. We will present them in section \ref{background:textsearch}.

% ML enabled data structures
The advancement of machine learning techniques leads to development of learned data structures, 
which aim to discover and exploit patterns in input data in order to achieve improvement in time and space efficiency.
One such example is learned indexes. Kraska et al. \cite{kraska2018case} showed that learned indexes could improve performance of speed up to 70\% 
when compared to cache-optimized B-Trees.

\section{Text Search}
\label{background:textsearch}

\subsection{Tokenization}

In natural language processing, tokenization is the process of splitting a long string into a list of tokens based on certain criteria.
One of the frequently used method is to split strings by whitespaces. For example, the sentence:
\begin{quote}
The quick brown fox jumps over the lazy dog
\end{quote}
will become a list of tokens as,
\begin{quote}
	The, quick, brown, fox, jumps, over, the, lazy, dog
\end{quote}

% char-level ngram

\subsection{Inverted index}
Inverted index is a data structure that allows fast full-text serach. It maps a search term, usually a word, to a list of documents that contain it. 




\subsection{Fuzzy matching}
\subsubsection{n-gram}
\subsubsection{Jaccard similarity}
\subsubsection{String distance} % edit distance/Levenshtein distance

% core concepts: inverted index, documents, fields, analysis(analyzers), searching(query, IndexSearcher)
% query syntax and examples
% TermQuery, PrefixQuery, WildcardQuery, PhraseQuery, FuzzyQuery
\subsection{Lucene}

\section{Machine Learning with Neural Networks}
\label{background:ml_nn}

{\color{red}
\begin{itemize}
    \item MLP
    \item Conv1D
    \item LSTM
    \item Attention
    \item Applications to NLP with token embeddings
\end{itemize}
}